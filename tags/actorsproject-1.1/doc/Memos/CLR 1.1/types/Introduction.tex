\chapter{Types in \Cal}\label{chap:TypesIntroduction}

This memo describes the use of {\em types} in our implementation of
the \Cal actor language \cite{CLR1}. The \Cal language itself is
largely type-agnostic, i.e. its specification makes minimal
assumptions about the existence and nature of data types. Any concrete
implementation must, of course, describe the nature of the data the
language manipulates, and how it allows users to describe it.

The way we use the term here, {\em data type} (or {\em type} for short) refers to a
collection of {\em (data) objects} or {\em values}.  If an object is a
member of the collection identified by a type, it is said to be an
{\em instance} of the type. Objects are the values of expressions, and
the contents of variables, and are characterized by the kinds of
operations that can be performed on them, and their behavior under
these operations.

Furthermore, in implementations, type information is also used to
determine how objects and variables are represented, including
important aspects of the representation such as the size (i.e. the
number of bits) required to store or communicate a data object of that
type.

Type information is also used to {\em statically check} the program
code in order to identify programming errors at during editing or at
compile time.



% Local Variables:
% TeX-master: "../CLR"
% End:


