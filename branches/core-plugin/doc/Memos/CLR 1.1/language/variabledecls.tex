

\chapter{Variables}\label{chap:Variables}

\index{expression!variable|(}
\index{variable|(}

\index{binding}
\index{variable!bound}
\index{variable!binding}
\sidenote{bindings}

Variables are placeholders for other values. They are said to be
{\em bound} to the value that they stand for. The association between
a variable and its value is called a {\em binding}.

\index{variable!binding!assignable}\index{binding!assignable}
\index{variable!binding!mutable}\index{binding!mutable}
\index{types!mutable}

\Cal \sidenote{assignable \& mutable bindings}distinguishes between
different kinds of bindings, depending on whether they can be assigned
to ({\em assignable variable binding}), and whether the object they
refer to may be mutated ({\em mutable variable binding}---cf. sections
\ref{sect:MutableTypes} and \ref{sect:IndexedAssignment}).

This chapter first explains how variables are {\em declared} inside \Cal
source code. It then proceeds to discuss the scoping rules of the
language, which govern the visibility of variables and also constrain
the kinds of declarations that are legal in \Cal.


\section{Variable declarations}\label{sect:VarDecls}

\index{variable!declaration}
\index{variable!declaration!explicit}
\index{variable!declaration!actor parameter}
\index{actor!parameter}
\index{parameter!actor}
\index{variable!declaration!input pattern}
\index{variable!declaration!closure parameter}
\index{closure!parameter}\index{parameter!closure}

\sidenote{declarations}

Each variable (with the exception of those predefined by the platform)
needs to be explicitly introduced before it can be used---it needs to
be {\em declared} or {\em imported} (section \ref{sect:Imports}). A
declaration determines the kind of binding associated with the
variable it declares, and potentially also its (variable) type.  There
are the following kinds of variable declarations:
\begin{itemize}
\item explicit variable declarations (section \ref{sect:ExplicitVarDecl},
\item actor parameters (chapter \ref{chap:Actor}),
\item input patterns (section \ref{sect:InputPatterns}),
\item parameters of a procedural or functional closure (section \ref{sect:Closures}).
\end{itemize}

Variables declared as actor parameters, in input patterns, or as
parameters of a procedural or functional closure are
neither assignable nor mutable.

The properties of a variable introduced by an explicit variable
declaration depend on the form of that declaration.

\subsection{Explicit variable declarations}\label{sect:ExplicitVarDecl}
\index{variable!declaration!explicit|(}

\index{variable!name}
\index{initialization expression} 
\index{variable!initialization expression}
\index{variable!type}

Syntactically, an explicit variable declaration\footnote{These
  declarations are called ``explicit'' to distinguish them from more
  ``implicit'' variable declarations that occur, e.g., in generators
  or input patterns.} looks as follows:

\index{mutable@\kwMutable}
\index{=@{\tt =}!in variable declarations}
\index{:=@{\tt :=}!in variable declarations}

\bgr
  VarDecl : [\kwMutable]~[Type]~ID ~[('=' | '\charColon=')
  ~Expression];
  | FunDecl | ProcDecl.
\egr

\index{declaration!of procedure}
\index{declaration!of function}
\index{function!declaration}
\index{procedure!declaration}

We will discuss function and procedure declarations ({\em FunDecl} and
{\em ProcDecl}) in section \ref{sect:FunProcDecl}.

An explicit variable declaration can take one of
the following forms, where {\tt T} is a type, {\tt v} an identifier
that is the variable name, and {\tt E} an expression of type {\tt T}:
\begin{itemize}
\item {\tt T v}---declares an assignable, non-mutable variable of type
  {\tt T} with the default value for that type as its initial
  value. It is an error for the type not to have a default value.
\item {\tt T v := E}---declares an assignable, non-mutable variable of
  type {\tt T} with the value of {\tt E} as its initial value.
\item {\tt mutable T v := E}---declares an assignable and mutable variable of
  type {\tt T} with the value of {\tt E} as its initial value.
\item {\tt mutable T v = E}---declares an non-assignable and mutable variable of
  type {\tt T} with the value of {\tt E} as its initial value.
\item {\tt T v = E}---declares a non-assignable, non-mutable variable of
  type {\tt T} with the value of {\tt E} as its initial value.
\end{itemize}

\index{variable!stateful}
\index{variable!stateless}
\sidenote{stateful vs stateless}
Variables declared in any of the first four ways are called {\em stateful
  variables}, because they or the object they are containing may be
changed by the execution of a statement. Variables declared in the
last way are referred to as {\em stateless variables}.

\index{var@\kwVar}
\index{let@\kwLet}
Explicit variable declarations may occur in the following places:
\begin{itemize}
\item actor state variables
\item the \kwVar block of a surrounding lexical context
\item variables introduced by a \kwLet-block
\end{itemize}

While actor state variables and variables introduced in a
\kwVar-block can be state variables as well as non-state variables, a
 \kwLet-block may only introduce non-state variables.

\index{variable!declaration!explicit|)}

\section{Variable scoping}\label{sect:VariableScoping}

\index{variable!scoping|(} \index{scope!of variable|see{variable, scoping}}

\index{variable!shadowing}
\index{variable!scoping!lexical}
\index{scoping!lexical}
\index{lexical scoping}

The scope of a variable is the lexical
construct that introduces it---all expressions and assignments using
its name inside this construct will refer to that variable binding,
\sidenote{lexical scoping}unless they occur inside some other construct that introduces a
 variable of the same name, in which case the inner variable shadows
the outer one.

\index{variable!initialization expression} 
\index{initialization expression!of a variable} 


In particular, this includes the \sidenote{initialization expression} {\em
  initialization expressions} that are used to compute the initial
values of the variables themselves. Consider e.g. the following group
of variable declarations inside the same construct, i.e. with the same
scope:
\begin{alltt}
  n = 1 + k,
  k = 6,
  m = k * n
\end{alltt}
\sidenote{well-formed declaration}
This set of declarations (of, in this case, non-mutable,
non-assignable variables, although this does not have a bearing on the rules for
initialization expression dependency) would lead to {\tt k} being set
to 6, {\tt n} to 7, and {\tt m} to 42. Initialization expressions may
not depend on each other in a circular manner---e.g., the following
list of variable declarations would not be well-formed:
\index{variable!declaration!not well-formed}
\begin{alltt}
  n = 1 + k,
  k = m - 36,
  m = k * n
\end{alltt}

\index{dependency set|(}\index{variable!dependency set|(}
\index{variable!dependency} \index{variable!free} 
\index{free variable}
\sidenote{dependency set}
More precisely, a variable may not be in its own {\em dependency set}.
Intuitively, this set contains all variables that need to be known in
order to compute the initialization expression. These are usually the
{\em free} variables of the expression itself, plus any free variables
used to compute them and so on---e.g., in the last example, {\tt k}
depended on {\tt m}, because {\tt m} is free in {\tt m - 36}, and
since {\tt m} in turn depends on {\tt k} and {\tt n}, and {\tt n} on
{\tt k}, the dependency set of {\tt k} is \{{\tt m}, {\tt k}, {\tt
  n}\}, which {\em does} contain {\tt k} itself and is therefore an
error.

This would suggest defining the dependency set as the transitive
closure of the free variable dependency relation---which would be a
much too strong criterion. Consider e.g. the following declaration:
\begin{alltt}
  f = lambda (n) :
    if n = 0 then 1 else n * f(n - 1) end
  end
\end{alltt}

\index{variable!dependency!circular}
\index{dependency!circular}
\index{recursive closure}\index{recursive function}
\index{function!recursive} \index{recursion|see{recursive closure}}
\sidenote{recursion}
Here, {\tt f} occurs free in the initialization expression of {\tt f},
which is clearly a circular dependency. Nevertheless, the above
definition simply describes a recursive function, and should thus be
admissible.

\index{closure!relation to variable dependency}
The reason why {\tt f} may occur free in its own definition without
causing a problem is that it occurs inside a closure---the {\em value}
of {\tt f} need not be known in order to construct the closure, as long
as it becomes known before we {\em use} it---i.e. before we actually
apply the closure to some argument.

We will now define the dependency sets $I_v$ and $D_v$ of a variable
$v$ among a set of variables $V$ that are defined simultaneously in
the same scope.

\index{variable!dependency set!definition@{\em definition}} 
\index{dependency set!definition@{\em definition}}
\index{dependency set!immediate}
\index{variable!dependency set!immediate}
\index{definition!dependency set}
\index{definition!immediate dependency set}
\begin{defn}[$I_v$, $D_v$---the dependency sets of a variable $v$]\label{def:DependencySets}
  Consider a set $V$ of variables $v$ which are defined
  simultaneously, i.e. the intial value of each of these variables
  defined by an expression $E_v$ which is in the scope of all the
  variables in $V$. Let $F_v$ be the set of free variables of
  $E_v$. As we are only interested in the free variables in $V$, we
  will usually use the intersection $F_v \cap V$. 
  
  The {\em dependency set} $D_v$ is defined as the smallest set
  such that the following holds:
\begin{eqnarray*}
\text{(a)}~&F_v \cap V \subseteq D_v \\
\text{(b)}~&\underset{x \in D_v}\bigcup D_x \subseteq D_v
\end{eqnarray*}

The\sidenote{immediate dependency set}
  {\em immediate dependency set} $I_v$ of each variable $v$ is defined
  as follows
\[I_v = \begin{cases}
  \emptyset \quad &\text{for} \quad E_v ~ \text{a closure}\\
  D_v \quad &\text{otherwise}
\end{cases}\]
\end{defn}

Intuitively, $D_v$ contains those variables in $V$ on which the object
bound to $v$ directly or indirectly depends. $I_v$ is the set of
variables whose values need to be known when the object computed by
$E_v$ is created---for most expressions, it is the same as $D_v$, but
for closures (procedural or functional) this set is empty, because
there is no need to evaluate the body of the closure in order to
construct the closure object.\footnote{Here we use the fact that
  closures can be constructed without the values of their free
  variables, which is clearly an artifact of the way we envision
  closures to be realized, but it is a useful one.}

Now we capture the notion of well-formedness of a set of
simultaneously defined variables $V$ as a condition on the dependency
sets as follows:

\index{declaration set!well-formed}
\index{variable!declaration set!well-formed}
\index{definition!well-formed declaration set}
\sidenote{well-formed declaration set}
\begin{defn}[Well-formed declaration set]
A set of simultaneously declared variables (a {\em declaration set}) $V$ is {\em well-formed} iff
for all $v\in V$
$$v \notin I_v$$
\end{defn}

Note that, as in the example above, a variable may occur free in its
own initialization expression, but still not be in its own immediate
dependency set, as this only includes those variables whose {\em
  value} must be known in order to compute the value of the declared variable.

This notion of well-formedness is useful because of the following property:

\index{corollary!mutual dependencies}
\begin{corollary}[No mutual dependencies in well-formed variable sets.]
  Given a well-formed variable set $V$, for any two variables $v_1,
  v_2 \in V$, we have the following property:
  $$\neg (v_1 \in I_{v_2} \wedge v_2 \in I_{v_1})$$
  That is, no two
  variables ever mutually immediately depend on each other.
\end{corollary}

The proof of this property is trivial, by contradiction and induction
over the definition of the dependency set (showing that mutual
dependency would entail self-dependency, and thus contradict
well-formedness).

\index{rationale!variable declaration!well-formedness}
\begin{rationale}
  Strictly speaking, this definition of well-formedness is a
  conservative approximation to what might be considered well-formed
  variable declarations. It is chosen so that (a) it allows commonly
  occurring circular dependencies to be expressed and (b) it can be
  implemented without undue efforts. But it is important to realize
  that it does declare some declaration sets as erroneous which could
  easily be interpreted. For instance, consider the following case:
  \begin{alltt}    f = lambda (x) :
      if x <= 1 then 1 else x * g(x - 1) end
    end,
    g = f
  \end{alltt}
  According to the definitions above, both {\tt g} ends up in its own
  immediate dependency set, and thus these declarations are not
  well-formed.
\end{rationale}


This allows us to construct the following relation over a set of
variables:

\index{variable!dependency relation}
\index{dependency relation}
\index{definition!dependency relation}
\sidenote{dependency order relation}
\begin{defn}[Dependency relation]
Given a set of variables $V$ defined in the same scope, we define a
relation $\prec$ on $V$ as follows:
$$v_1 \prec v_2 \Longleftrightarrow v_1 \in I_{v_2}$$
\end{defn}

\index{partial order!dependency relation is}
\index{dependency relation!is partial order}
In other words, a variable is `smaller' than another according to this
relation iff it occurs in its dependency set, i.e. iff it has to be
defined before the other can be defined. The well-formedness of the
declaration set implies that this relation is a non-reflexive partial
order, since variables may not mutually depend on each other.

This order allows us to execute variable declarations in such a way
that immediate dependencies are always evaluated before the dependent
variable is initialized.

\index{example!mutually recursive variable declarations|(}
\index{dependency set!example}
\index{variable!dependency set!example}
\begin{example}
Consider the following variable definitions:
\begin{alltt}
  a = f(x),
  f = lambda (v) : 
      if p(v) then v else g(h(v)) end
  end,
  g = lambda (w) : b * f(w) end,
  b = k
\end{alltt}

Note that {\tt f} and {\tt g} are mutually recursive.

\index{variable!free}
\index{free variable}
The following lists the immediate dependencies and the free variable
dependencies of each variable above,\footnote{We are disregarding here
  the implicit variable references that will be introduced when the
  operators are resolved to function calls---strictly speaking, they
  would become part of $F_v$, but as they are always referring to
  global variables, and would thus disappear from $F_v\cap V$ anyway,
  we do not bother with them in the example.} along with their
intersection with the set $\{a, f, g, b\}$, which is the set $V$ in
this case:

\begin{center}
\begin{tabular}[h]{c|c|c|c|c}
$v$   &    $F_v$   & $F_v \cap V$ & $D_v$  & $I_v$  \\
\hline\hline
$a$   &  $\{f, x\}$ & $\{f\}$   &  $\{f, g, b\}$ &  $\{f, g, b\}$  \\\hline
$f$   &  $\{p, g, h\}$ &  $\{g\}$  &  $\{f, g, b\}$  &  $\emptyset$ \\\hline
$g$   &  $\{b, f\}$ &  $\{b, f\}$  &  $\{f, g, b\}$ & $\emptyset$ \\\hline
$b$   &  $\{k\}$ & $\emptyset$   & $\emptyset$  & $\emptyset$ \\\hline
\end{tabular}
\end{center}\vspace{4mm}

Now let us compute the dependency set $D_a$ of the variable {\tt
  a}. We start with the set 
$$F_a \cap V = \{f\}$$
Now we compute 
$$(F_a \cup F_f) = \{f, g\}$$
Then
$$(F_a \cup F_f \cup F_g)\cap V = \{f, g, b\}$$
Finally, we reach a fixpoint at
$$D_a = (F_a \cup F_f \cup F_g \cup F_b)\cap V = \{f, g, b\}$$

Similarly, we compute $D_f$, $D_g$, and $D_b$. The immediate
dependency sets of $f$ and $g$ are empty, because their initialization
expressions are closures. The immediate dependency sets of $a$ and $b$
are the same as their dependency sets, which in the case of $b$ is
also empty, because it does not depend on any variable in $V$. As a
result of this analysis, we see that the variables {\tt f}, {\tt g},
and {\tt b} may be defined in any order, but all must be defined
before {\tt a}, as it depends on all of them.
\end{example}
\index{example!mutually recursive variable declarations|)}


\index{example!non-well-formed variable declarations|(}
\begin{example}
Now consider the following slightly changed variable definitions, with
an additional dependency added to {\tt b}:
\begin{alltt}
  a = f(x),
  f = lambda (v) :
      if p(v) then v else g(h(v)) end
  end,
  g = lambda (w) :
      b * f(w)
  end,
  b = a * k
\end{alltt}

Again, the following table lists the dependency sets:

\begin{center}
\begin{tabular}[h]{c|c|c|c|c}
$v$   &    $F_v$   & $F_v \cap V$ & $D_v$  & $I_v$ \\
\hline\hline
$a$   &  $\{f, x\}$ & $\{f\}$   &  $\{f, g, b, a\}$ & $\{f, g, b, a\}$ \\\hline
$f$   &  $\{p, g, h\}$ &  $\{g\}$   & $\{f, g, b, a\}$ & $\emptyset$ \\\hline
$g$   &  $\{b, f\}$ &  $\{b, f\}$   & $\{f, g, b, a\}$ & $\emptyset$ \\\hline
$b$   &  $\{a, k\}$ & $\{a\}$   &  $\{f, g, b, a\}$ & $\{f, g, b, a\}$ \\\hline
\end{tabular}
\end{center} \vspace{4mm}

Now, computing $D_a$ proceeds as follows:
$$F_a \cap V = \{f\}$$
$$(F_a \cup F_f)\cap V = \{f, g\}$$
$$(F_a \cup F_f \cup F_g)\cap V = \{f, g, b\}$$
$$(F_a \cup F_f \cup F_g \cup F_b)\cap V = \{f, g, b, a\}$$
$$D_a = (F_a \cup F_f \cup F_g \cup F_b \cup F_a)\cap V = \{f, g, b, a\}$$

Obviously, in this case $a \in I_a$, thus the set of variable
definitions is not well-formed.
\end{example}
\index{example!non-well-formed variable declarations|)}


\index{dependency set|)}\index{variable!dependency set|)}

\index{variable!scoping|)} 



\index{expression!variable|)}
\index{variable|)}



% Local Variables:
% TeX-master: "../reference"
% End:
