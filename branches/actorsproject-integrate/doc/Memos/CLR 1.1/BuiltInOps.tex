
\chapter{Basic runtime infrastructure}\label{app:Runtime}

This appendix describes the basic runtime infrastructure, i.e. the
kinds of objects and operations on them that implementations must
provide in order to implement \Cal. A realistic
implementation might support many more data types and have a much more
extensive library of functions and procedures. Most of the facilities described in
this appendix are used in \Cal constructs, such as collections in
generators, or maps and sets in input patterns and output expressions. 

\section{Operator symbols}\label{app:PredefOps}

\index{operator!basic|(}
\index{operator!predefined|(}

The following table summarizes the predefined unary operator symbols
in \Cal.

\index{operator!basic!unary}
\index{operator!predefined!unary}
\index{unary operator!predefined}

\begin{center}
\begin{tabular}[h]{|r|c|l|}
{\bf Operator}   &  {\bf Operand type} & {\bf Meaning}  \\
\hline\hline
{\tt not}   & {\tt Boolean} & logical negation \\\hline
{\tt \#}    & {\tt Collection[T]} & number of elements \\ \cline{2-3}
            & {\tt Map[K, V]} & number of mappings \\\hline
{\tt dom}   & {\tt Map[K, V]} & domain of a map \\\hline
{\tt rng}   & {\tt Map[K, V]} & range of a map \\\hline
{\tt -}     & {\tt Number} & arithmetic negation \\ \hline
\end{tabular}
\end{center}\vspace{3mm}

\index{operator!precedence}
\index{operator!binding strength}

The next table lists the predefined binary operator symbols in the
\Cal language. They are sorted by increasing binding strength. Their
binding strength is given by a precedence figure $P$, higher precedence
binds stronger.

\index{operator!basic!binary}
\index{operator!predefined!binary}
\index{binary operator!predefined}
\index{operator!symbols}

\begin{center}
\begin{tabular}[h]{|c|r|c|c|p{3.5cm}|}
{\bf P} & {\bf Operator}  & {\bf Operand 1} & {\bf Operand 2} & {\bf Meaning}  \\
\hline\hline
1 & {\tt and}  & {\tt Boolean} & {\tt Boolean} & logical conjunction \\ \cline{2-5}
  & {\tt or}   & {\tt Boolean} & {\tt Boolean} & logical disjunction \\ \hline

2  & {\tt =}     & {\tt Object} & {\tt Object} & equality \\ \cline{2-5}
   & {\tt !=}    & {\tt Object} & {\tt Object} & inequality \\ \cline{2-5}
   & {\tt <}     & {\tt Number} & {\tt Number} &  less than\\ \cline{3-5}
   &             & {\tt Set[T]} & {\tt Set[T]} &  \\ \cline{3-5}
            &    & {\tt String} & {\tt String} &  \\ \cline{3-5}
            &    & {\tt Character} & {\tt Character} &  \\ \cline{2-5}
   & {\tt <=}     & \multicolumn{2}{c|}{\em analogous to {\tt <}} &  less then or equal \\ \cline{2-5}
   & {\tt >}      & \multicolumn{2}{c|}{\em analogous to {\tt <}} &  greater than \\ \cline{2-5}
   & {\tt >=}     & \multicolumn{2}{c|}{\em analogous to {\tt <}} &  greater than or equal \\ \hline

3 & {\tt in}    & {\tt T} & {\tt Collection[T]} & membership \\\hline

4 & {\tt +}     & {\tt Number} & {\tt Number} &  addition \\ \cline{3-5}
  &           &  {\tt Set[T]} & {\tt Set[T]} & union \\ \cline{3-5}
  &           &  {\tt List[T]} & {\tt List[T]} & concatenation \\ \cline{3-5}
  &           &  {\tt Map[K, V]} & {\tt Map[K, V]} & map union \\ \cline{2-5}
  & {\tt -}     & {\tt Number} & {\tt Number} & difference \\ \cline{3-5}
  &           &  {\tt Set[T]} & {\tt Set[T]} & set difference \\ \hline

5 & {\tt div}   & {\tt Number} & {\tt Number} & integral division \\ \cline{2-5}
  & {\tt mod}   & {\tt Number} & {\tt Number} & modulo \\ \cline{2-5}
  & {\tt *}     & {\tt Number} & {\tt Number} & multiplication \\ \cline{3-5}
  &             & {\tt Set[T]} & {\tt Set} & intersection \\ \cline{2-5}
  & {\tt /}     & {\tt Number} & {\tt Number} & division \\ \hline

6 & $\widehat{\ }$  & {\tt Number} & {\tt Number} & exponentiation \\ \hline

\end{tabular}
\end{center}

\index{operator!predefined|)}
\index{operator!basic|)}


\section{Basic data types and their operations}\label{app:BasicTypes}

This section shortly lists the basic data types used by \Cal,
describes their fundamental properties, and the operations they can be
expected to support.

Many of the basic data types are {\em composite}, i.e. elements of
these types may be objects that contain other objects. In the cases
presented in this section, such a composite can be recursively
decomposed into a subobject and a residual composite, until a
canonical {\em empty} composite is reached. Functions operating on
such composites can often easily be described by
using\sidenote{selector function} {\em selector
  functions}. A selector function takes a composite and another
function as arguments. It picks one of the subobjects and passes it
and the residual composite into the argument function. In the cases
below, we can describe our composite data types by some properties of
the data objects and the behavior of the selector function.


\subsection{{\tt Collection[T]}---collections}\label{sect:Collections}

\index{Collection@{\tt Collection} (type)|(}

\index{selector function}

An object of type {\tt Collection[T]} is a finite collection of
objects of type {\tt T}. The most fundamental operation on it is {\em
  selecting} an {\em element}, which is really dividing the collection
into an element and a residual collection of the same kind.
\sidenote{{\tt selectf}}The {\tt selectf} selector function does this,
and has the following signature:

\index{selector function!of collection}
\index{selectf@{\tt selectf}!for collection}
\index{Collection@{\tt Collection} (type)!selector function}

\begin{alltt}  selectf<A, B>: [[A, Collection[A] --> B], 
                  Collection[A], 
                  B --> B]
\end{alltt}

Different kinds of collections may differ in the way the selector
function picks the element from the collection (e.g., in the case of
lists, it is guaranteed to pick the first element of the list, see
\ref{sect:Lists}).

The two \sidenote{size and membership}fundamental operations on a
collection are computing its {\em size} and testing whether a given
object is contained by it. These are represented by two operators, the
prefix operator {\tt \#} and the infix operator {\tt in}. We assume
that these are substituted by function calls as follows:

\index{Collection@{\tt Collection} (type)!size|(}
\index{size!of collection|(}
\index{Collection@{\tt Collection} (type)!membership test|(}
\index{membership test!in collection|(}
\index{in@{\tt in} (operator)!on collection|(}
\index{\#@{\tt \#}!on collection|(}

\begin{tabular}[h]{c@{~$\equiv$~}l}
  {\tt \# C} & {\tt \$size(C)}\\
  {\tt a in C} & {\tt \$member(a, C)}
\end{tabular}

We can define the {\tt \$size} and {\tt \$member} functions using the
{\tt selectf} function in the following manner:

\begin{alltt}function \$size(C) : // prefix \#
    selectf(lambda (a, C1) :
              1 + \$size(C1)
            end, C, 0)
end  

function \$member(x, C) :
    selectf(lambda (a, C1) :
                if a = x then true else 
                    \$member(x, C1)
                end
            end, C, false) 
end
\end{alltt}

\index{Collection@{\tt Collection} (type)!size|)}
\index{size!of collection|)}
\index{Collection@{\tt Collection} (type)!membership test|)}
\index{membership test!in collection|)}
\index{in@{\tt in} (operator)!on collection|)}
\index{\#@{\tt \#}!on collection|)}

\index{selector procedure!of collection}
\index{selectp@{\tt selectp}!definition}

Finally, we can define iteration over a collection by writing a
procedure that works somewhat analogous to the {\tt selectf} function:

\begin{alltt}procedure selectp(p, C) begin
    selectf(lambda (a, C1) : 
                proc () do p(a, C1); end 
            end, 
            C,
            proc () do end)();
    end 
end
\end{alltt}

Consequently, the signature of {\tt selectp} is as follows:

\begin{alltt}  selectp<A>: [[A, Collection[A] -->], Collection[A]]\end{alltt}

\subsubsection{Generator functions and procedures}

\index{\$mapadd@{\tt \$mapadd}!definition|(}

The selection function and procedure also allow us to construct the
various functions and procedures used in defining generators in
\Cal. For instance, the {\tt \$mapadd} function used in comprehensions
(see section \ref{sect:ComprehensionsWithGenerators}), which is slightly
different depending on whether it produces a set, a list, or a map:

\begin{alltt}function \$mapaddset (C, f) :
    selectf(lambda (a, C1) : f(a) + \$mapaddset(C1, f) end, 
            C, \{\})
end

function \$mapaddlist (C, f) :
    selectf(lambda (a, C1) : f(a) + \$mapaddlist(C1, f) end, 
            C, [])
end

function \$mapaddmap (M, f) :
    selectf(lambda (a, M1) : f(a) + \$mapaddmap(M1, f) end, 
            M, map \{\})
end\end{alltt}

\index{\$mapadd@{\tt \$mapadd}!definition|)}

\index{\$iterate@{\tt \$iterate}!definition|(}

The {\tt \$iterate} procedure used in the foreach-statement (see
section \ref{sect:ForeachStmt}) can be built on top of {\tt selectp}
as follows:

\begin{alltt}procedure \$iterate (C, p) begin
    selectp(proc (a, C1) do p(a); \$iterate(C1, p); end, C);
end
\end{alltt}

\index{\$iterate@{\tt \$iterate}!definition|)}

\index{\$try@{\tt \$try}!definition|(}

Finally, the {\tt \$try} procedure used in the choose-statement (see
section \ref{sect:NDChoice}) can be constructed as follows:

\begin{alltt}procedure \$try (C, f, p) begin
    selectp(proc (a, C1) do
                if f() then
                    p(a);
                    \$try(C1, f, p);
                end
            end, C);
end
\end{alltt}

\index{\$try@{\tt \$try}!definition|)}

\index{Collection@{\tt Collection} (type)|)}


\subsection{{\tt Seq[T]}---sequences}\label{sect:Sequences}

\index{Seq@{\tt Seq} (type)|(}

\index{Seq@{\tt Seq} (type)!index into}
\index{Seq@{\tt Seq} (type)!finite|see{{\tt List} (type)}}
\index{Seq@{\tt Seq} (type)!infinite}

Sequences are arrangements of objects that are indexed by non-negative
integers, starting at zero if not empty, and either ending at some maximal index
$k$, or not at all. The first kind of sequence is called {\em finite
  sequence} or {\em list} (see section \ref{sect:Lists}), the second
kind is called {\em infinite sequence}.

\index{selector function!of sequence}
\index{selectf@{\tt selectf}!for sequence}
\index{Seq@{\tt Seq} (type)!selector function}

The selector function of sequences splits the non-empty
sequence into its first element and the rest of the sequence:
\begin{alltt}  selectf<A, B>: [[A, Seq[A] --> B], 
                  Seq[A], 
                  B --> B]
\end{alltt}

\index{Seq@{\tt Seq} (type)!indexer!definition|(}
\index{Seq@{\tt Seq} (type)!hasElement@{\tt hasElement}!definition|(}
\index{hasElement@{\tt hasElement}!definition|(}

The\sidenote{{\tt hasElement}} only operations on sequences are
indexing (using a single integral index) and testing whether a given
non-negative integer is a valid index. They can be expressed in terms
of the selector as follows:
\begin{alltt}function \$nth(n, S) : // indexer
    selectf(lambda (a, R) :
                if n = 0 then a else
                    \$nth(n - 1, R)
                end, S, null)
end

function hasElement (n, S) :
    selectf(lambda (a, R) :
                if n = 0 then true else
                    hasElement(n - 1, R)
                end, S, false)
end
\end{alltt}

\index{Seq@{\tt Seq} (type)!indexer!definition|)}
\index{Seq@{\tt Seq} (type)!hasElement@{\tt hasElement}!definition|)}
\index{hasElement@{\tt hasElement}!definition|)}

\index{Seq@{\tt Seq} (type)|)}


\subsection{{\tt Set[T] < Collection[T]}---sets}\label{sect:Sets}

\index{Set@{\tt Set} (type)|(}

\index{selector function!of set}
\index{selectf@{\tt selectf}!for set}
\index{Set@{\tt Set} (type)!selector function}

Sets are a special kind of collection (see section
\ref{sect:Collections}) that guarantee that each object occurs in them
at most once. The selection function \sidenote{nondeterministic {\tt
    selectf}}{\tt selectf} is {\em nondeterministic} on sets with two
or more elements, i.e. it is unspecified which element will be chosen.

\index{Set@{\tt Set} (type)!subset order}
\index{subset order}

Sets\sidenote{{\tt subset}: partial order} are partially ordered by
the {\em subset} order, which can be defined as follows:
\begin{alltt}function subset(S1, S2) :
    selectf(lambda (a, S) :
                if a in S2 then
                    subset(S, S2)
                else
                    false
                end
            end, S1, true)
end\end{alltt}

\index{Set@{\tt Set} (type)!comparison operators}
\index{Set@{\tt Set} (type)!equality}

The\sidenote{set equality} comparison operators ({\tt <}, {\tt <=},
{\tt >}, {\tt >=}) are based on the subset order. In addition, two
sets are considered {\em equal} if they are subsets of each other,
i.e. they contain the same elements.

\index{Set@{\tt Set} (type)!construction}
\index{Set@{\tt Set} (type)!addElement@{\tt addElement}}

The\sidenote{\tt addElement} fundamental way of constructing sets is
to add one element to an existing set. This can be done using the
function {\tt addElement}, which has the following signature:
\begin{alltt}
  addElement<T> : [T, Set[T] --> Set[T]]
\end{alltt}
The result of {\tt addElement(a, S)} is the smallest set that contains
both, {\tt a} and all the elements in {\tt S}.

\index{Set@{\tt Set} (type)!union|(}
\index{Set@{\tt Set} (type)!intersection|(}
\index{Set@{\tt Set} (type)!difference|(}


Sets also have operators for union ({\tt +}), intersection ({\tt *})
and set difference ({\tt -}), which can be defined as follows:

\begin{alltt}function \$union (S1, S2) : // infix +
    selectf(lambda(a, S) :
                addElement(a, \$union(S, S2))
            end, S1, \{\})
end

function \$intersection(S1, S2) : // infix *
    selectf(lambda(a, S) :
                if a in S2 then
                    addElement(a, \$intersection(S, S2))
                else              
                    \$intersection(S, S2)
                end
            end, S1, \{\})
end

function \$setDifference(S1, S2) : // infix -
    selectf(lambda(a, S) :
                if a in S2 then
                    \$setDifference(S, S2)
                else
                    addElement(a, \$setDifference(S, S2))
                end
            end, S1, \{\})
end      
\end{alltt}

\index{Set@{\tt Set} (type)!union|)}
\index{Set@{\tt Set} (type)!intersection|)}
\index{Set@{\tt Set} (type)!difference|)}


In addition to these basic operations, there are set comprehensions
for computing sets (section \ref{sect:Comprehensions}).

\index{Set@{\tt Set} (type)|)}


\subsection{{\tt List[T] < Collection[T],
    Seq[T]}---lists}\label{sect:Lists}

\index{List@{\tt List} (type)|(}

\index{selector function!of list}
\index{selectf@{\tt selectf}!for list}
\index{List@{\tt List} (type)!selector function}

Lists are finite, sequential arrangements of objects. They are
collections and sequences at
the same time. As a consequence, its selector function splits the
(non-empty) list into its first element and the rest of the list.

\index{List@{\tt List} (type)!size}
\index{List@{\tt List} (type)!valid indices}
\index{List@{\tt List} (type)!hasElement@{\tt hasElement}}


For any list, the value computed by the {\tt \$size} function (section
\ref{sect:Collections}) defines the valid indices (via {\tt \$nth} and
{\tt hasElement}, see section \ref{sect:Sequences}) into the list, which
are the non-negative integers smaller than that number.

\index{List@{\tt List} (type)!constructor}
\index{List@{\tt List} (type)!cons@{\tt cons}}

Similar\sidenote{{\tt cons} constructor} to sets, lists can be
constructed elementwise using a {\tt cons} function, that adds an
element to the front of an existing list:
\begin{alltt}
  cons<T> : [T, List[T] --> List[T]]
\end{alltt}

\index{List@{\tt List} (type)!concatenation}

Lists can be concatenated using the {\tt +} operator, which can be
defined as follows:
\begin{alltt}function \$add (L1, L2) : // operator +
    selectf(lambda (a, L) :
                cons(a, \$add(L, L2))
            end, L1, L2)
end
\end{alltt}

Similarly to sets, list can be created using list comprehensions
(section \ref{sect:Comprehensions}).

\index{List@{\tt List} (type)|)}


\subsection{{\tt Map[K, V]}---maps}\label{sect:Maps}

\index{Map@{\tt Map} (type)|(}

\index{Map@{\tt Map} (type)!key}
\index{key!of map}
\index{Map@{\tt Map} (type)!value}
\index{value!of map}
\index{Map@{\tt Map} (type)!domain}
\index{domain!of map}
\index{Map@{\tt Map} (type)!range}
\index{range!of map}

A map of type {\tt Map[K, V]} maps finitely many {\em keys} of type {\tt K} to {\em values} of
type {\tt V}. The set of keys of a map is called its {\em domain},
while the set of values is called its {\em range}. 

\index{selector function!of map}
\index{selectf@{\tt selectf}!for map}
\index{Map@{\tt Map} (type)!selector function}

Like collections, maps are accessed using a selector
function. In the case of maps, this has the following signature:
\begin{alltt}selectf<K, V>: [[K, V, Map[K, V] --> A],
                Map[K, V],
                A --> A]\end{alltt}

It separates a non-empty map into a key, the corresponding value, and
the residual map, which it then applies its first argument to.

\index{Map@{\tt Map} (type)!domain!definition|(}
\index{domain!of map!definition|(}
\index{Map@{\tt Map} (type)!range!definition|(}
\index{range!of map!definition|(}

The domain and range of a map can then be defined like this:
\begin{alltt}function \$domain(M) : // operator dom
    selectf(lambda (k, v, M1) :
                addElement(k, \$domain(M1))
            end, M, \{\})
end

function \$range(M) : // operator rng
    selectf(lambda (k, v, M1) :
                addElement(v, \$range(M1))
            end, M, \{\})
end
\end{alltt}

\index{Map@{\tt Map} (type)!domain!definition|)}
\index{domain!of map!definition|)}
\index{Map@{\tt Map} (type)!range!definition|)}
\index{range!of map!definition|)}

\index{Map@{\tt Map} (type)!indexer}


The value corresponding to a key in a map is retrieved via an indexer,
using the key as in index into the map. The indexing function is
defined as follows:
\begin{alltt}function \$get(x, M) // indexer
    selectf(lambda(k, v, M1) :
                if x = k then v else
                    \$get(x, M1)
                end
            end, M, null)
end\end{alltt}


\index{Map@{\tt Map} (type)|)}

\subsection{{\tt Number}---numbers}\label{sect:Numbers}

\index{Number@{\tt Number} (type)|(}

Most of the number system of \Cal is left to
platform definition. The language itself provides two kinds of
numeric literals (integers are decimal fractions). There are a number
of operators on numbers, cf. section \ref{app:PredefOps} for details.


\index{Integer@{\tt Integer} (type)}

The type {\tt Number} contains the subtype {\tt Integer} of integer numbers.

\index{Number@{\tt Number} (type)|)}

\subsection{{\tt Character}---characters}

\index{Character@{\tt Character} (type)|(}

Characters are usually related to other data types, such as numbers
(via some encoding) or strings (as components of strings). The details
of this are left to the definition of platforms.

Characters are (possibly partially) ordered, and thus support the
relational operators {\tt <}, {\tt <=}, {\tt >}, {\tt >=}.

\index{Character@{\tt Character} (type)|)}


\subsection{{\tt String < List[Character]}---strings}

\index{String@{\tt String} (type)|(}

Strings are lists of characters, supported by a special syntax for
their construction. It is left to an implementation whether they are
identical to a corresponding list of characters, i.e. whether\\
\exindent{\tt "abc" = ['a', 'b', 'c']}

\index{String@{\tt String} (type)!order}
\index{lexicographical order}
\index{order!lexicographical}

Strings `inherit' the order from characters, and expand it
lexicographically:
\begin{alltt}function \$stringLE (s1, s2) : // operator <=
    selectf(lambda (c1, r1) :
                selectf(lambda (c2, r2) :
                            if c1 = c2 then
                                \$stringLE(r1, r2)
                            else 
                                c1 < c2
                            end
                        end, s2, false)
            end, s1, true)
end
\end{alltt}

\index{String@{\tt String} (type)|)}


\subsection{{\tt Boolean}---truth values}\label{sect:Booleans}

\index{Boolean@{\tt Boolean} (type)|(}
\index{true@\kwTrue}
\index{false@\kwFalse}

The {\tt Boolean} data type represents truth values. It is supported
by two constant literals (\kwTrue ~and \kwFalse) as well as a number
of operators, cf. section \ref{app:PredefOps}.


\index{Boolean@{\tt Boolean} (type)|)}


\subsection{{\tt Null}---the null value}\label{sect:Null}

\index{Null@{\tt Null} (type)|(}

The {\tt Null} data type has only one representative, the object
denoted by the literal \kwNull. Apart from equality tests, no
operations are defined on this type.

\index{Null@{\tt Null} (type)|)}


% Local Variables:
% TeX-master: "reference.tex"
% End:











