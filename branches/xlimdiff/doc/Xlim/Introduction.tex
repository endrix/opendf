
\chapter{Introduction}\label{chap:Introduction}

\section{Scope/Context}
This memo describes XLIM, an XML format for representing a
language independent model of imperative programs.

A program is {\em imperative} if it specifies an algorithm as a
sequence of statements that modify a collection of variables and
memory---as opposed to, e.g., a {\em functional} or {\em declarative}
description. Imperative programs typically contain assignments and
control structures such as loops and conditional statements, in
addition to expressions.

When building tools that process imperative code, such as compilers
that generate software and hardware implementations from it, it
becomes necessary to design a representation for that code that
facilitates the relevant analyses and transformations. The source code
itself is usually not very useful for this purpose, because many of
the common processing steps involve the dependency structure among
expressions and statements in the code. This dependency structure is
only implicit in the original source, and needs to be established by
matching variable names according to the scoping rules of the
particular language.

The abstract syntax tree (AST) form of the code, which is usually
generated by a parser, shares the same limitation. Even though it is
more structural than the plain text of the code, the structures in it
represent the lexical structure of the program, rather than the actual
dependencies between its elements.

XLIM, in addition to maintaining information about the hierarchical
lexical structure explicitly provided by the user, also directly
represents the dependency relation between the elements of a program.
It is essentially equivalent to a program in {\em static single assignment}
(SSA) form, in which each variable is assigned to in only one place of
the source.

\section{Format Overview}
XLIM is, firstly, an XML document containing a hierarchy of elements as
defined in chapter \ref{elementdefs}.  The elements which make up the
document define the functionality being captured.  {\it Operation} elements
define atomic operators while {\it modules} provide for grouping and
define control structures.  As with any XML document, the inclusion of
additional elements, not defined in this document, is allowed.  Those
tools/programs which consume XLIM may impose certain additional
required data to be included in the document.  The inclusion of additional
elements in an XLIM document does not make that document mal-formed
and should be tolerated by any consuming applications.

XLIM documents consist of a design containing 3 main types of elements.
First, the design contains an enumeration of port elements.  These define the
way data is transferred to and from the functionality represented.  Further,
internal ports may be used for communication between parallel top-level (modules)
functionality (Note that state variables may also be used to communicate between
parallel top-level modules).  Second, the design contains an enumeration of
program state (state variables).  These elements may be simple scalars or larger
'addressable' units.  Third, the design contains one or more units capturing the
functionality of the program.  Each unit is a collection of {\it modules}
and {\it operations} related by their hierarchy and connectivity (defined
by attributes on the operations). A design may have any number of ``ports'',
state elements, and parallel functional blocks.  



% Local Variables:
% TeX-master: "XLIM"
% End:




