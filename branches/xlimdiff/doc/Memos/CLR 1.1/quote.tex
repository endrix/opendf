
\markboth{}{}

\vspace*{2cm}

First, we want to establish the idea that a computer language is not
just a way of getting a computer to perform operations but rather that
it is a novel formal medium for expressing ideas about methodology.
Thus, programs must be written for people to read, and only
incidentally for machines to execute. Second, we believe that the
essential material to be addressed by a subject at this level is not
the syntax of particular programming-language constructs, nor clever
algorithms for computing particular functions efficiently, nor even
the mathematical analysis of algorithms and the foundations of
computing, but rather the techniques used to control the intellectual
complexity of large software systems.

[...]

Underlying our approach to this subject is our conviction that
``computer science'' is not a science and that its significance has
little to do with computers. The computer revolution is a revolution
in the way we think and in the way we express what we think. The
essence of this change is what might best be called {\em procedural
  epistemology}---the study of the structure of knowledge from an
imperative point of view, as opposed to the more declarative point of
view taken by classical mathematical subjects. Mathematics provides a
framework for dealing precisely with notions of "what is." Computation
provides a framework for dealing precisely with notions of ``how to''

\begin{flushright}
 Harold Abelson, Gerald Jay Sussman\\
Structure and Interpretation of Computer Programs \cite{AbelsonH+99}
\end{flushright}
\vspace{2cm}

What a thing means is simply what habits it involves [...] there is
no distinction of meaning so fine as to consist in anything but a
possible difference in practice.

\begin{flushright}
Charles S. Pierce \cite{PierceCS1878}
\end{flushright}
